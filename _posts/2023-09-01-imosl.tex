---
layout: post
title: Geometry 3
permalink: /imosl/
date: 2023-09-01 00:00:00 -0000
categories: maths
---
Một lời giải mang tính chất tính toán tỉ lệ của mình cho bài này, ngoài ra trên AoPS còn những lời giải khác như: sử dụng tính chất điểm Humpty, nghịch đảo hay dùng số phức.

https://artofproblemsolving.com/community/c6h1671271p25296276

Solution.

Let $M$ be the midpoint of $BC$ and $R$ be the center of the circumcircle of triangle $HPQ$
Claim 1: $AH$ is tangent to $\odot(HPQ)$
This is just angle chase: 
$$\angle AHP=\angle HAB+\angle HBA=90^0-\angle B+90^0-\angle A=\angle C$$
 and 
 $$\angle PQH=\angle HCA+\angle CAO=180^0-\angle A-\angle B=\angle C$$
 Claim 2: $\triangle HPQ \sim \triangle ABC (aa)$
Cause we have $\angle PQH=\angle C$, and similarly $\angle QPH=\angle B$
Therefore 
$$\dfrac{RH}{R_{ABC}}=\dfrac{HP}{AB}=\dfrac{HP}{HA}.\dfrac{HA}{AB}=\dfrac{\sin \angle HAO}{\sin \angle B}.\dfrac{AH}{AB}=\dfrac{\sin \angle B-\angle C}{\sin \angle B}.\dfrac{AH}{AB}$$
 We will prove that $\dfrac{HR}{HA}=\dfrac{DM}{DA}$ (then apply Thales' theorem we get the result), which is equivalent to
$$MD=\sin (\angle B-\angle C).R $$ $$\Leftrightarrow BC/2-AB.\cos B=\sin (\angle B-\angle C).R $$
$$\Leftrightarrow R.\sin A-2R.\sin C.\cos B=R.(\sin B.\cos C-\cos B.\sin C)$$
 which is true, Q.E.D!

![_config.yml]({{ site.baseurl }}/images/g3-imosl-2017.png)
